%-----------------------------------------------------------------------------------------------------------------------------------------------%
%	The MIT License (MIT)
%
%	Copyright (c) 2021 Jitin Nair
%
%	Permission is hereby granted, free of charge, to any person obtaining a copy
%	of this software and associated documentation files (the "Software"), to deal
%	in the Software without restriction, including without limitation the rights
%	to use, copy, modify, merge, publish, distribute, sublicense, and/or sell
%	copies of the Software, and to permit persons to whom the Software is
%	furnished to do so, subject to the following conditions:
%	
%	THE SOFTWARE IS PROVIDED "AS IS", WITHOUT WARRANTY OF ANY KIND, EXPRESS OR
%	IMPLIED, INCLUDING BUT NOT LIMITED TO THE WARRANTIES OF MERCHANTABILITY,
%	FITNESS FOR A PARTICULAR PURPOSE AND NONINFRINGEMENT. IN NO EVENT SHALL THE
%	AUTHORS OR COPYRIGHT HOLDERS BE LIABLE FOR ANY CLAIM, DAMAGES OR OTHER
%	LIABILITY, WHETHER IN AN ACTION OF CONTRACT, TORT OR OTHERWISE, ARISING FROM,
%	OUT OF OR IN CONNECTION WITH THE SOFTWARE OR THE USE OR OTHER DEALINGS IN
%	THE SOFTWARE.
%	
%
%-----------------------------------------------------------------------------------------------------------------------------------------------%

%----------------------------------------------------------------------------------------
%	DOCUMENT DEFINITION
%----------------------------------------------------------------------------------------

% article class because we want to fully customize the page and not use a cv template
\documentclass[a4paper,12pt]{article}

%----------------------------------------------------------------------------------------
%	FONT
%----------------------------------------------------------------------------------------

% % fontspec allows you to use TTF/OTF fonts directly
% \usepackage{fontspec}
% \defaultfontfeatures{Ligatures=TeX}

% % modified for ShareLaTeX use
% \setmainfont[
% SmallCapsFont = Fontin-SmallCaps.otf,
% BoldFont = Fontin-Bold.otf,
% ItalicFont = Fontin-Italic.otf
% ]
% {Fontin.otf}

%----------------------------------------------------------------------------------------
%	PACKAGES
%----------------------------------------------------------------------------------------
\usepackage{url}
\usepackage{parskip} 	

%other packages for formatting
\RequirePackage{color}
\RequirePackage{graphicx}
\usepackage[usenames,dvipsnames]{xcolor}
\usepackage[scale=0.9]{geometry}

%tabularx environment
\usepackage{tabularx}

%for lists within experience section
\usepackage{enumitem}

% centered version of 'X' col. type
\newcolumntype{C}{>{\centering\arraybackslash}X} 

%to prevent spillover of tabular into next pages
\usepackage{supertabular}
\usepackage{tabularx}
\newlength{\fullcollw}
\setlength{\fullcollw}{0.47\textwidth}

%custom \section
\usepackage{titlesec}				
\usepackage{multicol}
\usepackage{multirow}

%CV Sections inspired by: 
%http://stefano.italians.nl/archives/26
\titleformat{\section}{\Large\scshape\raggedright}{}{0em}{}[\titlerule]
\titlespacing{\section}{0pt}{10pt}{10pt}

%for publications
\usepackage[style=authoryear,sorting=ynt, maxbibnames=2]{biblatex}

%Setup hyperref package, and colours for links
\usepackage[unicode, draft=false]{hyperref}
\definecolor{linkcolour}{rgb}{0,0.2,0.6}
\hypersetup{colorlinks,breaklinks,urlcolor=linkcolour,linkcolor=linkcolour}
\addbibresource{citations.bib}
\setlength\bibitemsep{1em}

%for social icons
\usepackage{fontawesome5}

%debug page outer frames
%\usepackage{showframe}


% job listing environments
\newenvironment{jobshort}[2]
    {
    \begin{tabularx}{\linewidth}{@{}l X r@{}}
    \textbf{#1} & \hfill &  #2 \\[3.75pt]
    \end{tabularx}
    }
    {
    }

\newenvironment{joblong}[2]
    {
    \begin{tabularx}{\linewidth}{@{}l X r@{}}
    \textbf{#1} & \hfill &  #2 \\[3.75pt]
    \end{tabularx}
    \begin{minipage}[t]{\linewidth}
    \begin{itemize}[nosep,after=\strut, leftmargin=1em, itemsep=3pt,label=--]
    }
    {
    \end{itemize}
    \end{minipage}    
    }



%----------------------------------------------------------------------------------------
%	BEGIN DOCUMENT
%----------------------------------------------------------------------------------------
\begin{document}

% non-numbered pages
\pagestyle{empty} 

%----------------------------------------------------------------------------------------
%	TITLE
%----------------------------------------------------------------------------------------

% \begin{tabularx}{\linewidth}{ @{}X X@{} }
% \huge{Your Name}\vspace{2pt} & \hfill \emoji{incoming-envelope} email@email.com \\
% \raisebox{-0.05\height}\faGithub\ username \ | \
% \raisebox{-0.00\height}\faLinkedin\ username \ | \ \raisebox{-0.05\height}\faGlobe \ mysite.com  & \hfill \emoji{calling} number
% \end{tabularx}

\begin{tabularx}{\linewidth}{@{} C @{}}
\Huge{Egil Guting} \\[7.5pt]
\href{https://github.com/henegg-mir}{\raisebox{-0.05\height}\faGithub\ henegg-mir} \ $|$ \ 
\href{https://www.linkedin.com/in/egil-guting-7b7930381/}{\raisebox{-0.05\height}\faLinkedin\ Egil Guting} \ $|$ \ 
\href{mailto:egil@guting.se}{\raisebox{-0.05\height}\faEnvelope \ egil@guting.se} \ $|$ \ 
\href{tel:+46760344247}{\raisebox{-0.05\height}\faMobile \ +46760344247} \\
\end{tabularx}

%----------------------------------------------------------------------------------------
% EXPERIENCE SECTIONS
%----------------------------------------------------------------------------------------

%Experience
%\section{Work Experience}

%\begin{jobshort}{Designation}{Jan 2021 - present}
%long long line of blah blah that will wrap when the table fills the column width long long line of blah blah that will wrap when the table fills the column width long long line of blah blah that will wrap when the table fills the column width long long line of blah blah that will wrap when the table fills the column width
%\end{jobshort}
  
%Projects
\section{Projects}

\begin{tabularx}{\linewidth}{ @{}l r@{} }
\textbf{Game Jam: Sphelsylt 2025} & \hfill \href{https://github.com/Titanothere/Sphelsylt-LV1-LP4-2025}{Link to Game} \\[3.75pt]
\multicolumn{2}{@{}X@{}}{I scripted interactions between the visual elements and the core mechanics. Basically ensuring that upgrades gained from the interface led to actual changes in player character's abilities. Making the game gave me the experience of working together with people and making quick decisions since we made the biggest bulk of the game in the last 7 hours.}  \\
\end{tabularx}

%----------------------------------------------------------------------------------------
%	EDUCATION
%----------------------------------------------------------------------------------------

% Master courses relevant for master project
% Game Jam
\section{Education}
\begin{tabularx}{\linewidth}{@{}l X@{}}	
2021 - 2024 Bachelor's Degree in Computer Science at \textbf{Chalmers} \\
2024 - present Master's programme High-Performance Computing at \textbf{Chalmers}  \\
\end{tabularx}

%----------------------------------------------------------------------------------------
%	PUBLICATIONS
%----------------------------------------------------------------------------------------
\section{Publications}
\begin{refsection}[citations.bib]
\nocite{*}
\printbibliography[heading=none]
\end{refsection}

In my bachelors thesis, I wrote about sexuality and gender in games. We chose a couple of games and analyzed how they dealt with including a diverse set of differing identities. We then asked people in a case study their opinions about the games. Lastly, we used the feedback from the case study to create a prototype on a character creator. This gave me insight into how to go from a concept to a finalized product.

%----------------------------------------------------------------------------------------
%	SKILLS
%----------------------------------------------------------------------------------------
\section{Non-profit work}

\begin{jobshort}{Hobby theater}{early 2010s to 2021}
During my years before Chalmers I was active in hobby theater where I together with others practised and acted in plays. I got to learn how to improvise and also during the later years got the opportunity to create a play together with my friends. It opened up new ways of cooperation by adapting your responses in improv depending on what other people did, but also practicing scenes together was an important part of the teamwork. We could spend hours workshopping a scene just to ensure it was delivered correctly.
\end{jobshort}

\begin{jobshort}{Member of Chalmers Computer Science board game committee}{Spring 2022 to 2023}
Worked in the Chalmers Computer Science and Engineering board game committe called DLude where we arranged board game nights for our student division. 

We also participated in the student's reception through board game nights for both bachelor's and master's students, coordinated an event with other committees on the reception for students of all programs to try out board games and participate in various activitites. 

We made sure to have meetings regularly, about once a week for our last event, to coordinate and finalize decisions together early which was necessary for all bigger events since they often required communication between committees from different programs. It also helped with our regular board game nights because we had to ensure that there was enough food and that there were rooms available.
\end{jobshort}

\begin{jobshort}{Plot writer on LARPs}{2022 to 2025}
I have participated in LARPs where I wrote characters and story for the events. For the most part this entailed writing the plot direction of different characters to create a cohesive and combined narrative. It was an interesting and rewarding experience to work together and create worlds and stories, especially because you had access to other peoples experiences and perspectives. It was the work that was most reminiscient of doing theater together with my friends since the creation of all character plots was very collaborative. 
As a plot writer you interpreted and wrote stories for players, which were written independently. Therefore, this made feedback very important when you had to ensure that all characters followed an uniform and cohesive story, at the same time the scope was often too big for a single person to follow which made feedback even more important.
\end{jobshort}

\vfill
\center{\footnotesize Last updated: \today}

\end{document}
